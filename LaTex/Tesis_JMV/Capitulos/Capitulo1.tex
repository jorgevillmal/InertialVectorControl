
\chapter{Introducción}
\setcounter{page}{1}
\renewcommand{\thepage}{\arabic{page}}


%%%%%%%%%%%%%%%%%%%%%%%%%%%%%%%%%%%%%%%%%%%%%%%%%%%%%%%5
% Motivacion
%%%%%%%%%%%%%%%%%%%%%%%%%%%%%%%%%%%%%%%%%%%%%%%%%%%%%%%%%5
\section{Motivación}
En el vasto panorama de la investigación científica y tecnológica contemporánea, el control de vehículos aéreos no tripulados (UAVs) se ha consolidado como una de las áreas más prometedoras y dinámicas. Esta destacada posición no es mera coincidencia; es el resultado de una serie de avances tecnológicos, industriales y académicos que han convergido en las últimas dos décadas. La aspiración de lograr una autonomía total en estos vehículos no es meramente una ambición académica, sino que tiene profundas implicaciones para una amplia gama de aplicaciones en diversos campos, desde la agricultura y la logística hasta la vigilancia y el entretenimiento.

Dentro del espectro de UAVs, los vehículos de despegue y aterrizaje vertical (VTOL), especialmente los quadrotors, han emergido como objetos de estudio particularmente intrigantes. Su versatilidad inherente, combinada con los desafíos teóricos y prácticos que presentan, los convierte en candidatos ideales para la investigación avanzada. Sin embargo, es precisamente esta complejidad la que ha revelado una serie de brechas en la literatura existente.

A pesar de los avances teóricos significativos, una observación crítica revela que gran parte de la investigación previa opera bajo supuestos ideales, comúnmente distantes de las particularidades y obstáculos de los ambientes reales. Las perturbaciones ambientales, las mediciones ruidosas, la incertidumbre paramétrica y otros factores impredecibles pueden desafiar y, en ocasiones, invalidar los modelos teóricos.

En este contexto, el presente trabajo se propone como una contribución esencial al campo. No se limita a abordar los desafíos teóricos del control de UAVs, sino que se aventura con determinación en el mundo experimental, buscando validar, refinar y, si es necesario, redefinir los paradigmas existentes. Esta transición de la teoría a la experimentación no es un mero ejercicio académico, sino un requisito esencial para asegurar que los modelos y controladores propuestos sean sólidos, duraderos y, primordialmente, funcionales en contextos reales.

La motivación que impulsa este trabajo es, por lo tanto, multifacética. Por un lado, yace la aspiración de aportar de manera sustancial al cuerpo teórico del control de UAVs, introduciendo propuestas vanguardistas y enfrentando vacíos presentes. Por otro lado, hay un reconocimiento explícito de la importancia crítica de la validación experimental. En última instancia, este trabajo aspira a ser un puente entre la teoría y la práctica, reafirmando que la verdadera medida de cualquier avance teórico reside en su capacidad para enfrentar y superar los desafíos del mundo real.


%%%%%%%%%%%%%%%%%%%%%%%%%%%%%%%%%%%%%%%%%%%%%%%%%%%%%%%5
%   Antecedentes
%%%%%%%%%%%%%%%%%%%%%%%%%%%%%%%%%%%%%%%%%%%%%%%%%%%%%%%%%5
\section{Antecedentes}

El control de vehículos aéreos no tripulados (UAVs) ha sido objeto de intensa investigación en las últimas décadas. Estos sistemas, que alguna vez fueron relegados a aplicaciones militares o de nicho, han encontrado un lugar en una variedad de industrias, desde la agricultura hasta la entrega de paquetes y la cinematografía.

Los UAVs, en particular los de despegue y aterrizaje vertical (VTOL) como los quadrotors, han capturado la atención de los investigadores y entusiastas por igual debido a su versatilidad y capacidad para operar en entornos que otros vehículos aéreos podrían encontrar desafiantes. Sin embargo, con esta versatilidad viene una serie de desafíos únicos en términos de control y estabilidad, especialmente dada la naturaleza subactuada de estos vehículos.

A lo largo de los años, se han propuesto numerosas técnicas para abordar estos desafíos. El enfoque de empuje vectorizado ha sido una solución popular. Sin embargo, un aspecto esencial y emergente en la literatura reciente es la propuesta de utilizar mediciones vectoriales para determinar la actitud del UAV. Esta metodología, que se aleja de las técnicas tradicionales basadas en matrices de rotación, cuaterniones unitarios o ángulos de Euler, ha demostrado ser prometedora.

A pesar de estos avances, ha habido una tendencia en la literatura a operar bajo supuestos ideales, a menudo sin considerar las complejidades y desafíos de los entornos operativos reales. Las perturbaciones ambientales, las mediciones ruidosas y la incertidumbre paramétrica son solo algunos de los factores que pueden afectar la eficacia de los controladores teóricos en situaciones prácticas.

En este contexto, surge la necesidad de investigaciones que no solo aborden los desafíos teóricos asociados con el control de UAVs, sino que también consideren la transición de estos modelos teóricos al ámbito experimental. La validación en entornos reales es esencial para garantizar que los controladores propuestos sean robustos y aplicables en situaciones prácticas.
%%%%%%%%%%%%%%%%%%%%%%%%%%%%%%%%%%%%%%%%%%%%%%%%%%%%%%%5
%   Formulacion del problema
%%%%%%%%%%%%%%%%%%%%%%%%%%%%%%%%%%%%%%%%%%%%%%%%%%%%%%%%%5
\section{Formulación del problema}

%%%%%%%%%%%%%%%%%%%%%%%%%%%%%%%%%%%%%%%%%%%%%%%%%%%%%%%%%%%%%5
% Objetivos
%%%%%%%%%%%%%%%%%%%%%%%%%%%%%%%%%%%%%%%%%%%%%%%%%%%%%%%%5
\section{Objetivos}
El presente trabajo de tesis tiene como objetivos:
\begin{itemize}
  \item D
  \item P
  \item I
  \item E
\end{itemize}

%%%%%%%%%%%%%%%%%%%%%%%%%%%%%%%%%%%%%%%%%%%%%%%%%%%%%%%5
%   Contribuciones
%%%%%%%%%%%%%%%%%%%%%%%%%%%%%%%%%%%%%%%%%%%%%%%%%%%%%%%%%5
\section{Contribuciones}
\begin{itemize}
    \item Encontrar condiciones para obtener 
    \item P
    \item I
    \item E
  \end{itemize}

%%%%%%%%%%%%%%%%%%%%%%%%%%%%%%%%%%%%%%%%%%%%%%%%%%%%%%%5
% Organizacion de la tesis
%%%%%%%%%%%%%%%%%%%%%%%%%%%%%%%%%%%%%%%%%%%%%%%%%%%%%%%%%5
\section{Organización de la tesis}
El presente trabajo de tesis se encuentra dividido en cinco capítulos,
siendo el presente el que concierne a la introducción. Los siguientes cuatro
 se describen a continuación:\\

En el \textbf{Capítulo 2}, se presenta el marco teórico 

En el \textbf{Capítulo 3}, se presenta 
Para validar el funcionamiento del gemelo digital, en el \textbf{Capítulo 4} 

